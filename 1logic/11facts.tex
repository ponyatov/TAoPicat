\secrel{Facts\\Факты}

The simplest kind of statement is called a \term{fact}. Facts are a means of stating that a \term{relation} holds between objects. An example is

\ru{Самый простой вид утверждения называется \term{фактом}. Факты обозначают существование некоторого \term{отношения} между объектами. Примером является}

\lst{1logic/father.pl}{1logic/father.pl}

\lst{1logic/father.pi}{1logic/father.pi}

For an extensional relation that contains a large number of tuples, it is tedious to define such a
relation as a predicate with \cat\ pattern-matching rules. It is worse if the relation has multiple keys.

\ru{Для экстенсионального отношения, которое содержит большое количество кортежей, 
утомительно определять такое отношение как предикат \cat\ через правила сопоставления с образцом.
Хуже, если такое отношение имеет несколько ключей.}

In order to facilitate the definition of extensional relations, \cat\ allows the inclusion of 
\term{predicate facts}\ in the form $p(t_1,..,t_n)$ in predicate definitions. 

\ru{Чтобы упростить определение экстенсиональных отношений, \cat\ позволяет включить 
\term{предикатные факты}\ в форме $p(t_1, .., t_n)$ в определения предикатов.}

% Facts and rules cannot co-exist in predicate
% definitions and facts must be ground. A predicate definition that consists of facts can be preceded
% by an \term{index declaration} in the form

% \ru{Факты и правила не могут сосуществовать в определениях предикатов, а факты должны быть обоснованы.
% Определению предиката, состоящему из фактов, может предшествовать \term{декларация индекса} в форме}

% \[ index (M_{11}, M_{12},.., M_{1n}) ... (M_{m1}, M_{m2},.., M_{mn}) \]

% \noindent
% where each $M_{ij}$ is either $+$ (meaning \textit{indexed}) or $-$ (meaning \textit{not indexed}). If no index declaration
% is given, then \cat\ assumes that no arguments are indexed. Facts are translated into pattern-matching rules before they are compiled.

% \noindent
% \ru{где каждый $M_{ij}$ либо $+$ (что означает \textit{индексирован}), либо $-$ (что означает \textit{не индексирован}). 
% Если объявление индекса не задано, \cat\ предполагает, что ни один аргумент не проиндексирован. 
% Факты переводятся в правила сопоставления с образцом до их компиляции.}

\bigskip
The fact says that Abraham is the father of Isaac, or that the relation \verb`father` holds between the 
individuals named \verb`abraham` and \verb`isaac`. Another name for a relation is a \term{predicate}.
Names of individuals are known as \term{atoms}. 

\ru{Факт говорит о том, что Авраам является отцом Исаака, или что существует 
отношение \texttt{father}\ между именами \texttt{abraham}\ и \texttt{isaac}.
Другое имя для отношения\ --- это \term{предикат}. Имена перечисленных людей известны как \term{атомы}.}

\clearpage
Similarly, \verb`plus(2,3,5)` expresses the relation 
then 2 plus 3 is 5. The familiar \verb`plus` relation can be realized via a set of facts that defines 
the addition table. An initial segment fo the table is

\ru{Аналогично, \texttt{plus(2,3,5)}\ выражает отношение, в котором 2 плюс 3 равно 5. 
Знакомое отношения \texttt{plus}\ может быть реализовано с помощью набора фактов, 
определяющих таблицу сложения. Начальный сегмент таблицы}

\lst{1logic/plus.pl}{1logic/plus.pl}

\lst{1logic/plus.pi}{1logic/plus.pi}

\lst{1logic/plus.pi.log}{1logic/plus.pi.log}

In contrast with \pro, \cat\ has two types of predicates: \term{backtracking} and \term{non-backtracking}.
It also able to do calculations in imperative style with infix syntax, and don't need to unify with temp variables.

\ru{
В отличие от \pro, в \cat\ есть два типа предикатов: с \term{бэктрекингом} и без. 
Он также способен выполнять вычисления в императивном стиле с инфиксным синтаксисом и не требует \term{унификации} с временными переменными.
}

The source code sample shown above has two \verb`main` sections. 
The first one is \textit{backtracking}\ \verb`?=>` it first \term{matches} every \verb`plus` \term{pattern}
with \term{variables} bound to values, then \verb`writef` outputs current variables values, 
and do infix computation \verb`X+Y` to check the result.
The \verb`fail` statement breaks forward pass and triggers backtracking to the next predicate match.

\ru{Приведенный выше пример исходного кода состоит из двух секций \texttt{main}. 
Первая с возвратами\ \texttt{?=>} сначала \term{шаблон} \texttt{plus} \term{сопоставляется} к каждым фактом
с \term{привязкой} переменных к значениям, затем \texttt{writef} выводит текущие значения переменных
и выполняет инфиксное вычисление \texttt{X+Y}, чтобы проверить результат.
Выражение \texttt{fail}\ прерывает прямой проход вычисления, 
запускает возврат бэктрекинга и поиск следующего совпадения предиката.}

When backtracking passes all facts, it stops, and the second main runs without backtracking
and return true to notify that program run without errors.

\ru{Когда бэктрекинг проходит все факты, он останавливается, и выполняется вторая секция
\texttt{main}\ --- она возвращает \texttt{true}, сообщая, что программа отработала без ошибок.}

\bigskip
As \cat\ \textit{has no queries syntax} in source code files, we need to explicitly use the \verb`main` predicate to run the computation.

\ru{Поскольку у Picat \emph{нет синтаксиса запросов} в файлах исходного кода, нам нужно явно использовать предикат \texttt{main}\ для запуска вычислений.}